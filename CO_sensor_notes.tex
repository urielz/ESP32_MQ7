\documentclass{article}

    % General document formatting
    \usepackage[margin=0.7in]{geometry}
    \usepackage[parfill]{parskip}
    \usepackage[utf8]{inputenc}
    \usepackage[dvipsnames]{xcolor}
    \usepackage[hidelinks]{hyperref}
    
    % Related to math
    \usepackage{amsmath,amssymb,amsfonts,amsthm}

\begin{document}

\title{Carbon Monoxide sensor (MQ7 \& BME280 on an ESP32) controlled via ESPHome on Home Assistant}
\maketitle


\section*{MQ-7 sensor:}
According to the MQ-7 data sheet (\url{https://github.com/urielz/ESP32_MQ7/blob/master/MQ-7.pdf}
), the MQ-7 runs in cycles. The cycle consist of a measuring period of 90s at
1.4V follow by a heating period of 60s at 5.0v. During the low voltage (low
temperature) phase, CO is absorbed on the plate, producing meaningful data. 
During the heating phase (5V), absorbed CO and other compounds evaporate from
the sensor plate, essentially cleaning it for the next measurement period.

\section*{ESPHome configuration:} 
Use the ESPHome adc sensor to interface with the built-in ADC of the devboard.
For the ESP32 the pin range is: GPIO32 to GPIO39.\\\\

Example YAML:\\
{\color{Gray} \rule{\linewidth}{0.6mm} }

\begin{verbatim}
sensor:
  - platform: adc
    pin: A0
    name: "sensor name"
    update_interval: never
\end{verbatim}
{\color{Gray} \rule{\linewidth}{0.6mm} }\\\\


The never update interval on the ADC sensor example is needed due to the measurement cycle of the MQ-7 sensor. 
Therefore the updates need to be trigger manually.
One way to do it would be to create an interval component that:\\\\
(a) Enables power via a switch\\
(b) Wait a bit (add delay)\\
(c) Triggers ADC update (component.update)\\
(d) Wait again\\
(e) Disable PSU\\\\

\section*{Calibration:}
.
\section*{Temperature and Relative Humidity dependance:}
.
\section*{Resources:}
\begin{itemize}
    \item\url{https://esphome.io/components/sensor/adc.html} 
\end{itemize}

\end{document}